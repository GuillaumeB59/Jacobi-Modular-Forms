\documentclass[10pt,a4paper]{article}
\usepackage[utf8]{inputenc}
\usepackage{amsmath}
\usepackage{amsfonts}
\usepackage{amssymb}
\usepackage{hyperref}
\begin{document}


\title{Jacobi Modular Forms: notes from the Coursera lectures by Valery Gritsenko}
\author{Guillaume Bioche}
\maketitle

\tableofcontents

\section{Motivations}
Motivations for studying Jacobi modular forms are related to the study of \begin{itemize}
\item generating functions for the number of representations of integers by quadrating forms
\item partition functions
\item definition of modular forms
\item elliptization of Ramanujan $\Delta$-function.
\end{itemize}

\subsection{Lattices}
\textbf{Definition} $L$ integral even positive definite quadratic lattice iff it has the following properties: 
\begin{itemize}
\item lattice: $L$ free $\mathbb{Z}$ module $\simeq\mathbb{Z}^n, n>0$
\item integral: there exists a symmetric, bilinear pairing $L\times L\rightarrow\mathbb{Z}$
\item even: $(v,v)\in2\mathbb{Z},\forall v\in L$
\item positive definite: $(v,v)>0,\forall v\neq 0$
\end{itemize}
\subsection{The theta function}
Let $L$ be an integral even p.d. quadratic lattice, and $r_L(2n)=\#\{v\in L |(v,v)-2n\}$. Let $\theta_L(q)=\sum_{n\geq0}r_L(2n)q^n$ and $q=e^{2i\pi\tau}, \tau\in\mathbb{H}_1=\{x+iy|y>0\}$. We have $\theta(\tau+1)=\theta(\tau)$ from the definition of $q$, but there is an additional hidden symmetry: 
\[\theta(-\frac{1}{\tau}) = (\frac{\tau}{i})^{\frac{n}{2}}(\textrm{det}L)^{-\frac{1}{2}}\theta_{L^*}(\tau)\]
where $L^*=\{u\in L\otimes \mathbb{Q}|\forall v\in L, (u,v)\in\mathbb{Z}\}$ is the dual lattice of $L$.
\\
Let us assume that $L^*=L$, which is equivalent to having $\textrm{det}L=1$ (and implies $n\equiv 0[8]$). Then we have
\begin{eqnarray*}
\theta_L(-\frac{1}{\tau})&=&\tau^{\frac{n}{2}}\theta_L(\tau)\\
\theta_L(\tau+1)&=&\theta(\tau)
\end{eqnarray*}
$\theta$ is then called a modular form of weight $n$ over $SL_2(\mathbb{Z})$.

$SL_2(\mathbb{R})=\{\left(\begin{array}{cc}a&b\\c&d\end{array}\right)\in M_2(\mathbb{R})|\textrm{det}M=1\}$ acts on $\mathbb{H}_1$ with
\[M:\tau\rightarrow\frac{a\tau+b}{c\tau+d}\in\mathbb{H}_1\]
Note that $SL_2(\mathbb{Z})$ is generated by $T=\left(\begin{array}{cc}1&1\\0&1\end{array}\right)$ and $S=\left(\begin{array}{cc}0&-1\\1&0\end{array}\right)$.
\subsection{Modular and abelian transforms}
\subsubsection{The partition function}
Let $L$ be an even integer positive definite lattice, $\tau\in\mathbb{H}_1$ and $z\in L\otimes\mathbb{C}\simeq\mathbb{C}^n$. We pose
\[\theta_L(\tau,z)=\sum_{v\in L}e^{i\pi\big((v,v)\tau+2(v,z)\big)}\]
This is the generating function for the distributions of lattice points of given norm and fixed scalar product with $z$. $\theta_L(\tau,z)$ is holomorphic over $\mathbb{H}_1\times(L\otimes\mathbb{C})$. Its symmetries are given by:
\begin{eqnarray}
\theta_L\left(\frac{-1}{\tau},\frac{z}{\tau}\right)&=&\tau^{\frac{n}{2}}e^{i\pi\frac{(z,z)}{\tau}}\theta(\tau,z)\quad\textrm{if }L=L^*\\
\theta_L(\tau,z+\lambda\tau+\mu)&=&e^{-i\pi\big(((\lambda,\lambda)\tau+2(\lambda,z)\big)}\theta_L(\tau,z),\quad\lambda,\mu\in\mathbb{Z}
\end{eqnarray}
\textbf{Exercise 1:} Prove the identities (use Poisson summation).\\

\subsubsection{Pullbacks of the partition function}
if $z=0$ then $\theta_L(\tau,0)=\theta_L(\tau)$.\\
Let $u\in L,(u,u)=2m>0$ and $z=u.Z,Z\in\mathbb{C}$. Then
\[\theta_L(\tau, u.Z)=\sum_{v\in L}e^{i\pi\big((v,v)\tau+2(u,v)Z\big)}\overset{\Delta}{=}\theta_{L,u}(\tau,Z)\]
$\theta_{L,u}(\tau,Z)$ is the Jacobi $\theta$-series of Eichler and Zagier \cite{EZ}.\\

\subsubsection{Pullbacks of the $\theta$-functions}
\[\theta_{L,u}(\tau,Z)=\sum_{n\geq0,l\in\mathbb{Z}}r_{L,u}(n,l)q^nr^l,\quad r=e^{2i\pi Z}\]
\[r_{L,u}(n,l)=\#\{v\in L|(v,v)=2n,(u,v)=l\}\]

Let $E_8$ be the even integral lattice of determinant 1 such that $E_8=E_8^*$. If $u$ is a root of $E_8$ (i.e. $(u,u)=2$), then $u_{E_8}^\bot=E_7$. In this case, $r_{E_8,u}(n,0)=\#\{v\in E_8|(v,v)=2n,(u,v)=0\}=e_{E_7}(n)$ so the Eichler-Zagier $\theta$-function provides a tool to study $E_7$ via $E_8$, which is in turn simpler.\\

\subsubsection{Elliptisation of the Ramanujan $\theta$-function:}
\textbf{Definition}: a modular form of weight $k\in\mathbb{Z}$ w.r.t. the modular group $SL_2(\mathbb{Z})$ is a holomorphic function:
\begin{eqnarray*}
f:\mathbb{H}_1&\rightarrow&\mathbb{C}\textrm{   satisfies:}\\
f\left(\frac{a\tau+b}{c\tau+d}\right)&=&(c\tau+d)^kf(\tau)\qquad(M)\\
f(\tau)&=&\sum_{n\geq 0}a(n)e^{2i\pi n\tau}\qquad(F)\\
\end{eqnarray*}
$(M)\iff$ $f$ is periodic.\\
$(F)$: $f(\tau)$ is holomorphic at $i\infty$ $\iff f(q)$ holomorphic at $q=0$.\\
\\
\textbf{Notation:}: $M_k=M_k(SL_2(\mathbb{Z})$ is the space of modular forms of fixed weight $k$. At this stage we admit that $\textrm{dim}_\mathbb{C}M_k<+\infty$ $\forall k$.\\
\\
As per the definition in \cite{SerreCA} the Ramanujan $\Delta$ function is 
\[\Delta(\tau)=q\prod_{n\geq 1}(1-q^n)^{24}\in M_{12}^{\textrm{cusp}}\]
We can lift this function as follows:
\[G_{4,4}(\tau,Z)=q^4r^{-4}\prod_{n\geq 1}(1-q^{n-1}r)^8(1-q^nr^{-1})^8(1-q^n)^8\]
where $G_{4,4}$ is a Jacobi form of weight 4 and index 4.\\


\subsection{Exercises week 1: even integral lattices}

\section{Definition of Jacobi forms:}
The plan of the section is to state a first definition of the Jacobi modular forms. Then we will present the Jacobi modular group $\Gamma^J$. In particular we will see that 
\[SL_2(\mathbb{Z})\hookrightarrow\Gamma^J\hookrightarrow Sp_2(\mathbb{Z})\]
 Finally we will reach the second definition of Jacobi modular forms.

\subsection{First definition}
Let $\tau\in\mathbb{H}=\{\tau\in\mathbb{C},\Im\tau>0\}$, $z\in\mathbb{C}$ and $k\in\mathbb{Z}$ the weight of the form, $ m\in\mathbb{N}$ its index.

Consider a holomorphic function $\phi(\tau,z):\mathbb{H}_1\times\mathbb{C}\rightarrow\mathbb{C}$. 
We state the following equations for $\left(\begin{array}{cc} a & b \\ c & d \end{array}\right)\in SL_2(\mathbb{Z})$ and $\lambda,\mu\in\mathbb{Z}$:
\begin{eqnarray*}
\phi\left(\frac{a\tau+b}{c\tau+d},\frac{z}{c\tau+d}\right)&=&(c\tau+d)^ke^{2i\pi m\frac{cz^2}{c\tau+d}}\phi(\tau,z)\qquad(M)\\
\phi(\tau,z+\lambda\tau+\mu)&=&e^{-2i\pi m(\lambda^2\tau+2\lambda z)}\phi(\tau,z)\qquad (E)
\end{eqnarray*}
(M) is the modular equation, (E) the elliptic equation. The latter is close to a double periodicity of $\phi$ over the elliptic curve $\mathbb{C}/(\mathbb{Z}+\tau\mathbb{Z})$.\\
From these equations one obtains immediately $\phi(\tau+1,z)=\phi(\tau,z)$, and \\
$\phi(\tau,z+1)=\phi(\tau,z)$. So $\phi$ has Fourier expansion \[\phi(\tau,z)=\sum_{n,l\in\mathbb{Z}}a(n,l)e^{2i\pi(n\tau+lz)}\]
\\
\textbf{Definition: } $\phi$ is called a holomorphic (resp. cusp, resp. weak) Jacobi form of weight $k$ and index $m$ if $a(n,l)=0$ unless $4nm-l^2\geq0$ (resp. $4nm-l^2>0$, resp. $n\geq0$).\\
\textbf{Notation:}\\
 $J_{k,m}$ is the space of holomorphic Jacobi forms.\\
 $J_{k,m}^{\textrm{cusp}}$ is the space of holomorphic Jacobi cusp forms\\
$J_{k,m}^w$ is the space of weak Jacobi forms. One has:
\[J_{k,m}^{\textrm{cusp}}\subset J_{k,m}\subset J_{k,m}^w\]
We take for granted that $\textrm{dim}J_{k,m}^w<+\infty$\\

\textbf{Example:} Let $u\in L$, $L$ an even positive definite quadratic positive lattice. The Fourier expansion of the Jacobi modular form is given by definition as:
\[\theta_{L,U}(\tau,z)=\sum_{v\in L}e^{i\pi\big((v,v)\tau+2(u,v)z\big)}=\sum_{n,l\in\mathbb{Z}}r_{L,u}(n,l)e^{2i\pi(n\tau+lz)}\]
The Fourier coefficients are given by
\[r_{L,u}(n,l)=\#\{v\in L|(v,v)=2n,(u,v)=l\}\]
So we have
\[r_{L,u}(n,l)\neq0\implies4nm-l^2>0\implies\left(\begin{array}{cc}(v,v)&(u,v)\\(u,v)&(u,u)\end{array}\right)=\left(\begin{array}{cc}4n&l\\l&m\end{array}\right)\succ0\]
\subsection{A remark for the modular equation}
To comment about the curious transformation $z\rightarrow\frac{z}{c\tau+d}$ in the modular equation, one notes that for an elliptic curve the invariance by $z\rightarrow z+\lambda\tau+\mu$ for $\lambda, \mu\in\mathbb{Z}$ means that $z\in\mathbb{C}/(\mathbb{Z}\tau+\mathbb{Z})=C_\tau$. However, $C_\tau\simeq C_{M\langle\tau\rangle}$, with $M\langle\tau\rangle=\frac{a\tau+b}{c\tau+d}$, where $M=\left(\begin{array}{cc}a&b\\c&d\end{array}\right)\in SL_2(\mathbb{Z})$. But $C_{M\langle\tau\rangle}=\mathbb{Z}M\langle\tau\rangle+\mathbb{Z}=
\frac{(a\mathbb{Z}+b\mathbb{Z})\tau+(c\mathbb{Z}+d\mathbb{Z})}{c\tau+d}$. Since $(a,b)=(c,d)=1$, this simplifies as $\mathbb{Z}M\langle\tau\rangle+\mathbb{Z}=(\mathbb{Z}
+\mathbb{Z}\tau)/(c\tau+d)$.\\
This hints at a relation between Jacobi modular forms and the universal elliptic curve $SL_2(\mathbb{Z})\setminus\mathbb{H}_1\times\mathbb{C}/	(\mathbb{Z}\tau+\mathbb{Z})$.
\subsection{Motivation for the Jacobi modular group}
1) If we specialize the modular equation to $z=0$ we obtain
\[\phi(\frac{a\tau+b}{c\tau+d},0)=(c\tau+d)^k\phi(\tau,0)\]
So if $\phi\in J_{k,m}$:
\[\phi(\tau,0)=\sum_{n\in\mathbb{Z}}
\Big(\sum_{l\in\mathbb{Z}, 4nm-l^2\geq0}a(n,l)e^{2i\pi n\tau}\Big)\]
Since $m\geq0$, $\phi(\tau,0)=\sum_{n\geq0}c(n)q^n,q=e^{2i\pi\tau}$. So $\phi(\tau,z)\in J_{k,m}\implies\phi(\tau,0)\in M_k\left(SL_2(\mathbb{Z})\right)$.\\
\\
\textbf{Question 1: lifting of modular forms} 
For $f\in M_k(SL_2(\mathbb{Z}))$, can we find $\phi\in J_{k,m}$ such that $\phi(\tau,0)=f$? We will see later that the answer is positive.\\
\\
\textbf{Question 2: other specializations.} Are $\phi(\tau,\frac{1}{2}),\phi(\tau,\frac{\tau+1}{2}),\phi(\tau,\frac{a\tau+b}{N})$ for $a, b \in \mathbb{Z}/N\mathbb{Z}$ modular forms ?
This leads to another definition of Jacobi forms. 
\\And finally, 
\\
\textbf{Question 3:} What is the Jacobi modular group ?\\
\section{The Jacobi modular group}
\subsection{The slash operator}
Let $M\in SL_2(\mathbb{Z})$, $f:\mathbb{H}_1\rightarrow\mathbb{C}$. Let us define $f_{|k}M(\tau)=(c\tau+d)^{-k}f(\tau)$ called the "slash $k$" operator. We remind that $j(M,\tau)=(c\tau+d)$ is the automorphic factor. Then one has
\[f_{|k}(M_1M_2)=(f_{|k}M_1)_{|k}M_2\]
\textbf{Proof:} One needs to show first that $M_1M_2\langle\tau\rangle=M_1\langle M_2\langle\tau\rangle\rangle$, and then that $j(M_1M_2,\tau)=j(M_1,M_2\langle\tau\rangle)j(M_2,\tau)$, the so-called "cocycle" condition. Cf exercises.\\
\\
\textbf{Definition:} A modular form of weight $k\in\mathbb{Z}$ w.r.t. $SL_2(\mathbb{Z})$ is a holomorphic function $f:\mathbb{H}\rightarrow\mathbb{C}$ such that $f_{|k}(M)=f\,,\forall	M\in SL_2(\mathbb{Z})$, and such that $f$ is holomorphic at $i\infty$.\\
\textbf{Question:} How can one give a definition of Jacobi modular forms in a similar way? To draw a parallel:
\begin{eqnarray*}
\Gamma^J&\dashrightarrow& SL_2(\mathbb{Z})\\
._{|k,m}&\dashrightarrow& ._{|k}\\
\textrm{s.t. }\phi_{|k,m}g&=&\phi\qquad\forall g\in\Gamma^J
\end{eqnarray*}
\subsection{$Sp_n(\mathbb{Z})$ as a generalization of $SL_2(\mathbb{Z})$}
\[Sp_n(\mathbb{Z})=\{M\in M_2(\mathbb{Z})|^tM\left(\begin{array}{cc}0&-E_n\\E_n&0\end{array}\right)M=\left(\begin{array}{cc}0&-E_n\\E_n&0\end{array}\right)\]
\textbf{Exercise 1:} Prove that $Sp_1(\mathbb{Z})=SL_2(\mathbb{Z})$.\\
\textbf{Exercise 2:}\\
a) Prove that $\left(\begin{array}{cc}A&0\\0&D\end{array}\right)\in Sp_n(\mathbb{Z})\iff D\in GL_n(\mathbb{Z})\,,A=\;^tD^{-1}$\\
b) Prove that $\left(\begin{array}{cc}E_n&B\\0&E_n\end{array}\right)\in Sp_n(\mathbb{Z})\iff \,^tB=B\in M_n(\mathbb{Z})$.
\subsection{The Siegel modular group}
$Sp_n(\mathbb{R})$ acts on the Siegel upper half-plane\\ $\mathbb{H}_n=\{Z=X+iY|X,Y\in M_n(\mathbb{R}), X=\,^tX\,,Y=\,^tY\,,Y>0\}$. Note that $\mathbb{H}_{n=1}=\mathbb{H}_1$ so notations are consistent with the previous lecture.\\
\\
$Sp_n(\mathbb{R})$ acts on $\mathbb{H}$: let $\left(\begin{array}{cc}A&B\\C&D\end{array}\right)\in Sp_n(\mathbb{R})$. Define \[M\langle Z\rangle=(AX+B)(CZ+D)^{-1}\]
Exercises will put emphasis on this group action. We obtain for $F:\mathbb{H}_n\rightarrow\mathbb{C}, M\in Sp_n(\mathbb{R}), k\in\mathbb{Z}$:
\begin{eqnarray*}
(F_{|k}M)(Z)&=&\textrm{det}(CZ+D)^{-k}F(M\langle Z\rangle)\\
F_{|k}M_1M_2&=&(F_{|k}M_1)_{|k}M_2
\end{eqnarray*}

\subsection{Siegel modular forms}
\textbf{Definition: } A Siegel modular form of weight $k$ w.r.t. $Sp_n(\mathbb{Z}),n\geq 2$ is a holomorphic function $F:\mathbb{H}_n\rightarrow\mathbb{C}$ such that $F_{|k}M=F$, $\forall M\in Sp_n(\mathbb{Z})$.
\textbf{Remark: }There is no condition on the Fourier expansion for $n\geq2$.

\subsection{The Jacobi modular group}
The Jacobi modular group is 
\[\Gamma^J=\left\{\left(\begin{array}{cccc}
*&1&*&*\\ *&1&*&*\\ *&0&*&*\\0&0&0&1\end{array}\right)\in Sp_2(\mathbb{Z})\right\}\]
Checks are made in the exercises. 
$\Gamma^J$ is a (parabolic) subgroup of $Sp_2(\mathbb{Z})$. Also, \begin{eqnarray*}
SL_2(\mathbb{Z}&\hookrightarrow&\Gamma^J\\
\left(\begin{array}{cc}a&b\\c&d\end{array}\right)&\rightarrow&\left[\left(\begin{array}{cc}a&b\\c&d\end{array}\right)\right]=\left(\begin{array}{cccc} a&0&b&0\\0&1&0&0\\c&0&d&0\\0&0&0&1\end{array}\right)
\end{eqnarray*}
And
\[M=\left(\begin{array}{cccc}
a&1&b&*\\ *&1&*&*\\ c&0&d&*\\0&0&0&1\end{array}\right)\implies\left[\left(\begin{array}{cc}a&b\\c&d\end{array}\right)\right]^{-1}M=\left(\begin{array}{cccc} 1&0&0&p\\-q&1&p&q\\0&0&1&r\\0&0&0&1\end{array}\right)\in\Gamma^J\subset Sp_2(\mathbb{Z})\]
The Heisenberg group is a subgroup $H(\mathbb{Z})<\Gamma^J$ described by
\[H(\mathbb{Z}=\left\{\left[\binom{p}{q},r\right]=\left(\begin{array}{cccc} 1&0&0&p\\-q&1&p&q\\0&0&1&r\\0&0&0&1\end{array}\right)|p,q,r\in\mathbb{Z}
\right\}\]
It represents the subgroup of unipotent matrices in $Sp_2(\mathbb{Z})$.\\
\\
\textbf{The second definition of Jacobi forms}:\\
$\mathbb{H}_2=\left\{Z=\left(\begin{array}{cc}\tau&z\\z&\omega\end{array}\right)\in M_2(\mathbb{C}|\Im Z>0\right\}$. We remark that $\Im Z>0\iff \tau,\omega\in\mathbb{H}_1,\Im\tau.\Im\omega-(\Im z)^2>0$. So $\forall (\tau,z)\in\mathbb{H}\times\mathbb{C},\exists\omega\in\mathbb{H}_1$ such that $\left(\begin{array}{cc}\tau&z\\z&\omega\end{array}\right)\in\mathbb{H}_2$.\\
For $\phi\in J_{k,m}$ let us define $\tilde{\phi}_m(\tau,z)\triangleq\phi(\tau,z)e^{2i\pi m\omega}$, where $m$ is called the index.\\
\\
Then $\phi$ satisfies both the modular and the elliptic equations iff:
\[\tilde{\phi}_m:\mathbb{H}_2\rightarrow\mathbb{C}:\tilde{\phi}_{m|k}g=\tilde{\phi}_m\forall g\in\Gamma^j\]

\subsection{The subgroups of the Jacobi modular group}
As shown above, we have $SL_2(\mathbb{Z})<Sp_2(\mathbb{Z})$. In particular, $\forall g\in Sp_2(\mathbb{Z}),\exists M\in SL_2(\mathbb{Z})$ s.t. 
\[[M]^{-1}g=\left(\begin{array}{cccc}
1&*&0&* \\ *&1&*&* \\ 0&0&1&* \\ 0&0&0&1
\end{array}\right)=\left(\begin{array}{cc}A & B \\ C & D\end{array}\right)\]
Using the symplectic relations, one obtains constraints between the blocks which lead to
\[[M]^{-1}g=\left(\begin{array}{cccc}
1&0&0&p \\ -q&1&p&r \\ 0&0&1&q \\ 0&0&0&1
\end{array}\right)=\left[\binom{p}{q},r\right]\]
Thus we obtain
\[\Gamma	^J=\left[SL_2(\mathbb{Z})\right].H(\mathbb{Z})\]
\textbf{Properties of $H(\mathbb{Z})$:}\\
1) We have
\[\left[\binom{p}{q},r\right].\left[\binom{p'}{q'},r'\right]=\left[\binom{p+p'}{q+q'},r+r'+\left|\begin{array}{cc}
p&p' \\ q & q'\end{array}\right|\right]\]
This immediately implies that $H(\mathbb{Z})$ is not commutative.\\
2) From above we deduce
\[\left[\binom{p}{q},r\right]^{-1}=\left[\binom{-p}{-q},-r\right]\]
3) The commutators are described by
\[h.h'.h^{-1}h'^{-1}=\left[\binom{0}{0},2\left|\begin{array}{cc}
p&p' \\ q & q'\end{array}\right|\right]\]
4) The center of $H(\mathbb{Z})$ is
\[Z\big(H(\mathbb{Z})\big)=\left\{\left[\binom{0}{0},r\right],\, r\in\mathbb{Z}\right\}\]
5)We have the exact sequence
\begin{eqnarray*}
\mathbb{Z}&\longrightarrow H(\mathbb{Z})\longrightarrow&\mathbb{Z}\times\mathbb{Z}
\longrightarrow 0\\
r&\longrightarrow\left[\binom{p}{q},r\right]\longrightarrow&\binom{p}{q}\in\mathbb{Z}^2
\end{eqnarray*}
\\
The Heisenberg group is thus the central extension of $\mathbb{Z}^2$.\\
\textbf{As exercises:}\\
\\
6) Let $V_H:H(\mathbb{Z})\rightarrow\left\{\pm1\right\}$, $V_H\left(\left[\binom{p}{q},r\right]\right)=(-1)^{p+q+pq+r}$.\\
Prove that $V_H$ is a character. What is its kernel ?\\
\\
7) Prove that $SL_2(\mathbb{Z})$ acts on $H(\mathbb{Z})$ by conjugation with :
\[[M]\left[\binom{p}{q},r\right][M]^{-1}=\left[M\binom{p}{q},r\right]\]
\\
8) Prove that $H(\mathbb{Z})\triangleleft\Gamma^J$ and that $\Gamma	^J = SL_2(\mathbb{Z})\ltimes H(\mathbb{Z})$.

\subsection{The action of $\Gamma^J$ on the Siegel upper half-plane $\mathbb{H}_2$}
Given the decomposition $\Gamma	^J = SL_2(\mathbb{Z})\ltimes H(\mathbb{Z})$ we can study the action of $\Gamma^J$ via the action of its subgroups. Let $Z=\left(\begin{array}{cc}\tau&z \\ z & \omega\end{array}\right)\in\mathbb{H}_2$, $M=\left(\begin{array}{cc}a&b \\c & d\end{array}\right)$. Then \[[M]\langle Z\rangle=\left(\begin{array}{cc}
\frac{a\tau+b}{c\tau+d} & \frac{z}{c\tau+d}\\ \frac{z}{c\tau+d}  & \omega-\frac{cz^2}{c\tau+d}\end{array}\right)\]
this corresponds to the transformation of the equation $(M_J)$: \[\big(\phi(\tau,z)e^{2i\pi m\omega}\big)_{|k}[M]=(c\tau+d)^{-k}e^{-2i\pi m\frac{cz^2}{c\tau+d}}\phi\left(\frac{a\tau+b}{c\tau+d},\frac{z}{c\tau+d}\right)e^{2i\pi m\omega}\]
\\
Regarding the action of $H(\mathbb{Z})$, let $h=\left[\binom{p}{-q},r\right]\in H(\mathbb{Z})$. Then we have
\begin{eqnarray*}
h\langle Z\rangle&=&\left(\begin{array}{cccc}
1&0&0&p\\ q&1&p&r\\ 0&0&1& -q \\ 0 &0 &0&1\end{array}\right)\langle\left(\begin{array}{cc}\tau&z \\ z & \omega\end{array}\right)\rangle\\
&=&\left(\begin{array}{cc}\tau&q\tau+z+p \\ q\tau+z+p& q^2\tau+2qz+qp + \omega+r\end{array}\right)
\end{eqnarray*}
We can compare the previous expression to the elliptic equation:
\[\phi(\tau,z){e^{2im\pi\omega}}_{|k}h=e^{2i\pi m(q^2\tau+2qz+pq+r)}\phi(\tau,z+q\tau+p)e^{2im\pi\omega}\]
\\
\textbf{The $\Gamma^J$ action on Jacobi forms}\\
\textbf{Definition} Let $\phi:\mathbb{H}_1\times\mathbb{C}\rightarrow\mathbb{C}$. We note that $\forall(\tau,z)\in\mathbb{H}_1\times\mathbb{C}$, $\exists\omega:\left(\begin{array}{cc}\tau &z \\ z &\omega	\end{array}\right)\in\mathbb{H}_2$. Let $g\in\Gamma^J(\mathbb{R})$. Let us pose
\[\phi_{|k,m}g\triangleq\Big(\big(\phi(\tau,z)e^{2i\pi m\omega}\big)_{|k}g\Big)e^{-2i\pi m\omega}\]
We obtain that:\\
1) $\phi_{|k,m}g$ depends only on $\tau$ and $z$,\\
2) $\phi_{|k,m}g_1g_2=(\phi_{|k,m}g_1)_{k,m}g_2$ is the right action of the real Jacobi group. \\
We remind the modular equation for the Jacobi forms:
\[\big(\phi(\tau,z)e^{2i\pi m\omega}\big)_{|k}[M]=(c\tau+d)^{-k}e^{-2i\pi m\frac{cz^2}{c\tau+d}}\phi\left(\frac{a\tau+b}{c\tau+d},\frac{z}{c\tau+d}\right)e^{2i\pi m\omega}\]
For $h=\left[\binom{p}{-q},r\right]\in H(\mathbb{R})$, we have
\begin{eqnarray*}
h\langle Z\rangle&=&\left(\begin{array}{cccc}1&0&0&p\\q&1&p&r\\0&0&1&-q\\0&0&0&1
\end{array}\right)\langle\left(\begin{array}{cc}\tau &z\\z&\omega\end{array}\right)\rangle\\
&=&\left(\begin{array}{cc}\tau&z+p\\q\tau+z+p&qz+\omega+r\end{array}\right)\left(\begin{array}{cc}1&-q\\0&1\end{array}\right)\\
&=&\left(\begin{array}{cc}\tau&q\tau+z+p\\q\tau+z+p&q^2\tau+2qz+qp+\omega+r\end{array}\right)
\end{eqnarray*}
So 
\[\phi(\tau,z){e^{2i\pi m\omega}}_{|k}h=e^{2i\pi m(q^2\tau+2qz+pq+r)}\phi(\tau,z+q\tau+p)e^{2i\pi m \omega}\]

\subsection{The second definition of Jacobi modular forms:}
\textbf{Definition:}\\
A holomorphic function $\phi:\mathbb{H}_1\times\mathbb{C}$ is a Jacobi modular form of weight $k$ and index $m\in\mathbb{Z}_{\geq0}$ if $\forall g\in\Gamma^J$: $\phi_{|k,m}g=\phi$. Moreover, $\phi(\tau,z)$ has a "good" Fourier expansion 
\[\phi(\tau,z)=\sum_{4nm-l^2\geq0}a(n,l)e^{2i\pi(n\tau+lz)}\]
i.e. $a(n,l)=0$ unless $4nm-l^2\geq0$.\\
\\
Note that we can write the Fourier expansion as:
\[\phi(\tau,z)e^{2im\pi\omega}\sum_{N\geq0}a\left(\left(\begin{array}{cc}n&l/2\\l/2&m\end{array}\right)\right)e^{2i\pi\textrm{Tr}(N.Z)},\,z=\left(\begin{array}{cc}\tau&z\\z&\omega\end{array}\right)\in\mathbb{H}_2
\]
A holomorphic Jacobi form is holomorphic function on $\mathbb{H}_2$ which is modular with respect to the parabolic subgroup $\Gamma^J<Sp_2(\mathbb{Z})$. We have thus the following illustration of the structure of modular forms:
\begin{eqnarray*}
SL_2(\mathbb{Z})\hookrightarrow&\Gamma^J&\hookrightarrow Sp_2
(\mathbb{Z})\\
\textrm{modular forms}\rightarrow&\textrm{Jacobi forms}&\rightarrow\textrm{Siegel modular forms}
\end{eqnarray*}

\section{Special values of Jacobi forms}
\textbf{Problem:} how can we analyze the modular properties of $\phi(\tau,\frac{1}{2}), \phi(\tau,\frac{\tau+1}{2})$ or $\phi(\tau,q\tau+p)$ for $q,p\in\mathbb{Q}$ where $\phi\in J_{k,m}$ of for $z$ any point of finite order in $\mathbb{C}/\mathbb{Z}\tau+\mathbb{Z}$ ? For this we will study the $_{|k,m}$-action of $\Gamma^J$.\\
Let $X=\binom{p}{-q}\in\mathbb{Q}^2$. Then 
\[\phi_{|k,m}[X,0]=e^{2i\pi m(q^2\tau+2qz+pq)}\phi(\tau,z+q\tau+p)\]
Specialising to $z=0$ we obtain
\[\phi_{|k,m}[X,0]=e^{2i\pi m(q^2\tau+pq)}\phi(\tau,q\tau+p)\]
Let us put $\Gamma_X=\left\{M\in SL_2(\mathbb{Z})|MX\equiv X[\mathbb{Z}^2]\right\}<SL_2(\mathbb{Z})$.
We have $\Gamma(N)<\Gamma_X$, where $\Gamma(N)=\left\{M\in SL_2(\mathbb{Z})|MX\equiv E_2[N]\right\}$ and $N$ is the minimum positive integer such that $N.X\in\mathbb{Z}^2$. $N$ is called the level of $X$.
\\
\\
\textbf{Exercise} Check that $\Gamma(N)<\Gamma_X$.
\\
\\
\textbf{Theorem:} Let $\phi\in J_{k,m}$, $X=\binom{p}{-q}\in\mathbb{Q}^2$. Then the function $\phi_X(\tau)=e^{2i\pi m(q^2\tau+pq)}\phi(\tau,q\tau+p)$ is a modular form of weight $k$ with respect to $\Gamma_X$ with a character $\chi_X(M)$:
\[\chi_X(M)=e^{2i\pi\textrm{det}(MX,X)}\]
\textbf{Proof:}
\\
Let us investigate the behaviour of $\phi_{|k,m}[X,0]$ under $_{|k,m}[M],M\in SL_2(\mathbb{Z})$. We have $[X,0].[M]=[M]\left([M]^{-1}.[X,0].[M]\right)$. Since $\phi\in J_{k,m}$, $\phi_{|k,m}[M]=\phi$. Then
\[\big(\phi_{k,m}[X,0]\big)_{k}[M]=\phi_{k,m}[M^{-1}X,0],\quad\forall M\in SL_2(\mathbb{Z})\]
Let $M\in\Gamma_X:MX-X\in\mathbb{Z}^2$. Then
\begin{eqnarray*}
[M^{-1}X,0]&=&=[M^{-1}X,0].[X,0]^{-1}.[X,0]\\
&=&[M^{-1}X,0].[-X,0].[X,0]\\
&=&[M^{-1}X-X,-\textrm{det}(M^{-1}X,X)].[X,0]\\
&=&[M^{-1}X-X,-\textrm{det}(X,MX)].[X,0]\\
\end{eqnarray*}
So we obtain for $\phi$ the transformation equation 
\[\phi_{|k,m}[M^{-1}X,0]=e^{2i\pi m\textrm{det}(MX,X)}\phi_{|k,m}[X,0]\]
We have thus proved the modular property for the pullback $\phi_X(\tau)$:
\[(\phi_{|k,m}[X,0])_{|k}[M]=e^{2i\pi m\textrm{det}(MX,X)}\phi_{|k,m}[X,0]\triangleq\chi_X(M)\phi_{|k,m}[X,0]\]
, where $M\rightarrow\chi_X(M)$ is a character of $\Gamma_X$. As a result, we obtain for $z=0$:
\[\phi_{X|k}M=e^{2i\pi m\textrm{det}(MX,X)}\phi_X\quad\forall M\in\Gamma_X\]
We check the Fourier expansion for the modular form.
\begin{eqnarray*}
\phi_X(\tau)&=&e^{2i\pi m(q^2\tau+pq)}\phi(\tau,q\tau+p)\\
&=&e^{2i\pi m(q^2\tau+pq)}\sum_{4nm-l^2\geq0}a(n,l)e^{2i\pi(n\tau+l(q\tau+p)}\\
&=&e^{2i\pi mpq}\sum_{4nm-l^2\geq0} a(n,l)e^{2i\pi lp}e^{2i\pi(mq^2+lq+n)\tau}
\end{eqnarray*}
$N=mq^2+lq+n\geq0$ since its discriminant is $\Delta=l^2-4nm\leq0$. So $\phi$ is holomorphic at $i\infty$.
To finish the proof, we have to study the Fourier expansion at all cusps: $\phi_X|M\,\forall M\in SL_2(\mathbb{Z})$.
\[\phi_{X|k,}M=\phi_{|k,m}[M^{-1}X,0])_{z=0}\]

Note that $M^{-1}X=\binom{p'}{q'}\in\mathbb{Q}^2$. This thus leads to a similar calculation. So one concludes that $(\phi_{X|k}M)(\tau)=\sum_{N\geq0}c(N)e^{2i\pi N\tau}$ and so that $\phi_X(\tau)$ is holomorphic on all cusps of $\Gamma_X$.
\\
\\
\textbf{Remark:} $a(n,l)=0$ unless $4nm-l^2\geq0$ $\iff\forall X\in\mathbb{Q}^2:$ $\phi_X(\tau)$ is holomorphic at $i\infty$. The interpretation is that $\phi(\tau)$ is the pullback, the second one corresponds to Siegel modular forms. XXXXXXXXXXXXXXXXXXXXXXXXXXXXXXXXXX
%week 4
\section{The zeros of elliptic functions}
\textbf{Theorem :} Let $\phi:\mathbb{C}\rightarrow\mathbb{C}$ be a holomorphic function, $m\in\mathbb{Z},\tau\in\mathbb{H}_1$ - considered fixed here. We assume that
\[\phi(z+\lambda\tau+\mu)=e^{-2im\pi(\lambda^2\tau+2\lambda z)}\phi(z)\quad\forall\lambda,\mu\in\mathbb{Z}\qquad\textrm{(E)}\]
Then if $\phi$ is not identically 0, then $\phi$ has exactly $2m$ zeros in any fundamental domain $\mathbb{C}/\mathbb{Z}\tau+\mathbb{Z}$, counting multiplicities.\\
\\
\textbf{Proof: }
\\
Let $D_{z_0}=\{z_0+x+iy|x,y\in[0,1[\,\}$ be a model of $\mathbb{C}/\mathbb{Z}\tau+\mathbb{Z}$. By varying $z_0$ we can make sure that $\phi(z)$ does not vanish on the boundary $\partial D_{z_0}$. The integral $\int_{\partial D_{z_0}}\frac{\phi'(z)}{\phi(z)}dz$ is equal to the number of zeroes of $\phi(z)$ in the fundamental domain. Let us consider the parallelepiped $z_0\overset{A}{\rightarrow }z_0+1\overset{B}{\rightarrow} z_0+1+\tau\overset{C}{\rightarrow} z_0+\tau\overset{D}{\rightarrow} z_0$.
\\
From (E) we get $\frac{\phi'(z+1)}{\phi(z+1)}=\frac{\phi'(z)}{\phi(z)}$. 
Moreover, $\frac{\phi'(z+\tau)}{\phi(z+\tau)}=\frac{\phi'(z)}{\phi(z)}-4im\pi$. So we obtain, taking into account the orientations:
\[\int_C\frac{\phi'(z)}{\phi(z)}dz=-\int_A\frac{\phi'(z)}{\phi(z)}dz+4im\pi\]
\[\int_B\frac{\phi'(z)}{\phi(z)}dz=-\int_D\frac{\phi'(z)}{\phi(z)}dz\]

So
\[\frac{1}{2i\pi}\int_{\partial D_{z_0}}\frac{\phi'(z)}{\phi(z)}dz=\frac{1}{2i\pi}\int_{ABCD}\frac{\phi'(z)}{\phi(z)}dz=2m\]


\subsection{Applications:}
1) If $\phi$ is meromorphic, then $2m$ is the number of zeros minus the number of poles.\\
\\
2) If $\phi$ is holomorphic, then $2m\geq0$ - this is why Jacobi forms are defined only for $m\geq0$.\\
\\
3) The function $\phi$ from the theorem is determined up to a constant by its zeroes:\\
\textbf{Proof:} Let us assume that $\exists\phi, \psi$ s.t. $\phi(z_1)=\psi(z_1)=0$, $\ldots$, $\phi(z_{2m}=\psi(z_{2m})=0$. So $\forall z_0\notin\{z_1,\ldots,z_{2m}\}$, $\phi(z_0)\psi(z_0)\neq0$.\\
So, using the theorem, $\phi(z)\psi(z_0)-\phi(z_0)\psi(z)$ has at least $2m+1$ zeroes, so vanishes identically. Hence we get $\psi(z)=\frac{\psi(z_0)}{\phi(z_0)}\phi(z)$.
\\
\\
4) $\phi\in J_{k,m}^w$, $z\rightarrow\phi(\tau,z)$ satisfies the conditions of the theorem. So $\phi(\tau,z)$ has exactly $2m$ zeroes in any fundamental domain 
$\mathbb{C}/\mathbb{Z}\tau+\mathbb{Z}$.\\
\\
5) \textbf{Case of $\phi\in J_{k,0}^w$, for $m=0$}: $\phi(\tau,z)-\phi(\tau,0)$ has a zero for $z=0$ So $\phi(\tau,z)=\phi(\tau,0)\in M_k$. So
$J_{k,0}\subseteq J_{k,0}^w=M_k\subseteq J_{k,0}$ and hence $J_{k,0}=J_k^w=M_k$.
\\
\\
\textbf{Theorem: } dim $J_{k,m}<+\infty$.
\\
\\
\textbf{Proof: }
$\forall X=\binom{p}{-q}\in\mathbb{Q}^2,\,\forall\phi\in J_{k,m}$, then $\phi_X(\tau)\triangleq e^{2i\pi(q^2\tau+pq)}\phi(\tau,q\tau+p)\in M_k(\Gamma_X,\chi_X)$, with $\Gamma_X\in SL_2(\mathbb{Z})$. Then the map
\[\phi(\tau,z)\rightarrow\big(\phi_{X_i}(\tau)\big)_{i=1}^{2m}\,,\,X_i\neq X_j\in\mathbb{Q}^2\]
is injective (because of corollary 3 above: $\phi$ is determined uniquely up to $\mathbb{C}^\times$ by its zeros).
So 
\[\textrm{dim}J_{k,m}\leq\sum_{i=1}^{2m}\textrm{dim} M_{X_i}(\Gamma_{X_i},\chi_{X_i})<+\infty\]
We will see below that this is a crude upper bound which can be improved upon.
\subsection{Taylor expansion of Jacobi forms}
Let $\phi(\tau,z)=\sum_{4nm-l^2\geq0}a(n,l)q^nr^l\in J_{k,m}$, with $q=e^{2i\pi\tau}$ and $r=e^{2i\pi z}$. So
\[\phi(\tau,z)=\sum_{d\geq d_0\geq0}f_d(\tau)z^d,\]
where $d_0=\textrm{ord}_{|z=0}\phi(\tau,z)$. We have \[f_0(\tau)=\phi(\tau,0)=\sum_{n\geq0}\sum_{l,4nm-l^2\geq0}a(n,l)q^n\in M_k(SL_2\big(\mathbb{Z})\big)\]
Also, since $\phi(\tau+1,z)=\phi(\tau,z)$, then 
$f_d(\tau+1)=f_d(\tau)$, so $f_d(\tau)=\sum_{n\geq0}c(n)q^n$.
\\
\\
We have the following property: if $d>0$, then $f_d(\tau)=\sum_{n>0}c(n)q^n$: there is no constant term since $a(n,l)=0$ for $n=0,\forall l\neq0$. Any $z^d$ term for $d>0$ must thus come from $n>0$. So if $d>0$, $f_d(\tau)=\sum_{n>0}c(n)q^n$, however $f_d$ is not in general a modular form.\\
\\
\textbf{Proposition: } $\phi\in J_{k,m}$ is determined uniquely by its first $(2m+1)$ Taylor coefficients $\big(f_0(\tau),\ldots,f_{2m}(\tau)\big)$.\\
\\
\textbf{Proof: } Let $\phi, \psi$ share the same first Taylor coefficients at order $0,\ldots, 2m$. Then ord$_{|z=0}(\phi-\psi)\geq 2m+1$, and since $\phi-\psi$ is of order $m$ it must vanish identically, hence $\phi\equiv\psi$.
\\
\\
As a side remark, one notes that for $\phi\in J_{k,m}$, since $-E_2\in SL_2(\mathbb{Z})$, $\phi(\tau,-z)=(-1)^k\phi(\tau,z)$. So
\[\phi(\tau,z)=\sum_{d\equiv k[2],d\geq0}f_d(\tau)z^d\]
\\
\textbf{Proposition: } For $\phi\in J_{k,m}$, $f_0(\tau)=\phi(\tau,0)\in M_k$. If $\phi(\tau,0)\equiv0$, then $f_{d_0}(\tau)\in S_{k+d_0}$ - a cusp form.\\
\\
\textbf{Proof: } $\forall M=\left(\begin{array}{cc}a&b\\c&d
\end{array}\right)\in SL_2(\mathbb{Z})$, we have
\[\phi(\frac{a\tau+b}{c\tau+d},\frac{z}{c\tau+d})=(c\tau+d)^ke^{2i\pi m\frac{cz^2}{c\tau+d}}\phi(\tau,z)\]
Expanding both sides, we get:
\[\sum_{d'\geq0}f_{d'}(M\langle\tau\rangle)\frac{z^{d'}}{(c\tau+d)^{d'}}=(c\tau+d)^ke^{2i\pi m\frac{cz^2}{c\tau+d}}\sum_{d'\geq0}f_{d'}(\tau)z^{d'}\]
Getting the full series requires the Taylor expansion of $e^{2i\pi m\frac{cz^2}{c\tau+d}}$. However, if $d_0=\textrm{ord}_{|z=0}\phi(\tau,z)$, then we obtain:
\[f_{d_0}(M\langle\tau\rangle)=(c\tau+d)^{k+d_0}f_{d_0}(\tau)\]
So $f_{d_0}(\tau)\in M_{k+d_0}$. Moreover, we see that $f_{d_0}\in S_{k+d_0}$ if $d_0>0$ since there is no constant term.\\
\\
\textbf{Applications: }
We want to understand the structure behind dim $J_{k,m}$.\\
1) $J_{2,1}=\{0\}$.\\
\\
2) We have: dim $J_{4,1}\leq1$, dim $J_{6,1}\leq1$, dim $J_{8,1}\leq1$.\\
\textbf{Proof:}\\
Let $\phi(\tau,z)\in J_{2k,1}$. Then $\phi(\tau,z)=f_0(\tau)+f_2(\tau)z^2+f_4(\tau)z^4+\ldots$\\ From the proposition before, $f_0(\tau)\in M_{2k}$, and after \cite{SerreCA}, we know that $M_2k=\mathbb{C}E_{2k}(\tau)$ for $2k=4,6,8,10$. If $f_0(\tau)\equiv0$, then $f_2(\tau)$ has to be a cusp form of weight $2k+2$, but $S_{2k+2}=\{0\}$, $2k+2\leq10$. So for weights $2k=2,4,6,8$, $\phi(\tau,z)\in J_{2k,1}$ is determined by $f_0(\tau)$.\\
\\
3) We have dim$J_{10,1}^{\textrm{cusp}}\leq1$,\quad dim$J_{8,2}^{\textrm{cusp}}\leq1$,\quad dim$J_{6,3}^{\textrm{cusp}}\leq1$\\
\\
\textbf{Proof for $J_{6,3}^{\textrm{cusp}}$:}
Similarly, $\phi(\tau,z)=f_0(\tau)+f_2(\tau)z^2+f_4(\tau)z^4+f_6(\tau,z)
z^6+\ldots$\\
Since $\phi$ is a cusp form and $S_6=\{0\}$, then:\\
$f_2\in S_8=\{0\}\implies f_4\in S_{10}=\{0\}\implies f_6(\tau)\in S_{12}=\mathbb{C}\Delta(\tau)$ where $\Delta(\tau)$ is the Ramanujan $\Delta$ function. We obtain that $\phi_{6,3}$ is determined uniquely by $f_6(\tau)\implies$ dim $J_{6,3}^\textrm{cusp}\leq1$.\\
The same argument provides the estimates for $J_{10,1}^\textrm{cusp}\leq1$ and $J_{8,2}^\textrm{cusp}$.\\
\\
In the next section, we are going to prove that dim $J_{6,3}^\textrm{cusp}=$ dim $J_{8,2}^\textrm{cusp}=$\\ dim $J_{10,1}^\textrm{cusp}=1$.

\section{The Jacobi $\theta$-series and Dedekind $\eta$ function}

The Dedkind $\eta$ function is defined as:
\begin{eqnarray*}
\eta:\mathbb{H}_1&\rightarrow&\mathbb{C}\\
\tau&\rightarrow&\eta(\tau)=q^{\frac{1}{24}}\prod_{n\geq1}(1-q^n)
\end{eqnarray*}
Its properties are:
\[\eta(\tau+1)=\eta(\tau)\]
\[\eta\left(\frac{-1}{\tau}\right)=\sqrt{\frac{\tau}{i}}\eta(\tau)\]
where the branch of the square root is taken s.t. $\sqrt{\tau}>0$ if $\tau\in\mathbb{R}^+$.
We have also:
\[\eta\left(\frac{a\tau+b}{c\tau+d}\right)=v_\eta(M)(c\tau+d)^\frac{1}{2}\eta(\tau),\quad \forall M=\left(\begin{array}{cc}a&b\\c&d\end{array}\right)\in SL_2(\mathbb{Z})\]
where $v_\eta(M)^{24}=1$, and $v_\eta(M_1)v_\eta(M_2)=\pm v_\eta(M_1M_2)$ is a projective character. $v_\eta$ is called a multiplier system of the modular group $SL_2(\mathbb{Z})$.\\ $v_\eta^2:SL_2(\mathbb{Z})\rightarrow\mathbb{C}$ is a character of order 12. We also have\\
\[\eta(\tau)^{2k}\in M_k\big(SL_2(\mathbb{Z}), (v_\eta^2)^k\big)\quad, \forall k\geq1\]
\[\eta(\tau)^{24}=q\prod_{n\geq1}(1-q^n)^{24}=\Delta(\tau)\in S_{12}\big(SL_2(\mathbb{Z})\big)\]
The following two identities are due to Euler:
\[\eta(\tau)=\sum_{n\geq1}\left(\frac{-12}{n}\right)q^{n^2/24}\]
where 
\begin{eqnarray*}
\left(\frac{12}{n}\right)&=&1\qquad\textrm{if }n\equiv\pm1 [12]\\
&=&-1\qquad\textrm{if }n\equiv\pm5 [12]\\
&=&0\qquad\textrm{if }(n,12)\neq1\\
\end{eqnarray*}
Also,
\[\eta(\tau)^3=\sum_{n\geq1}\big(\frac{-4}{n}\big)q^{n^2/8}\]
\begin{eqnarray*}
\left(\frac{-4}{n}\right)&=&\pm1\qquad\textrm{if }n\equiv\pm1 [4]\\
&=&0\qquad\textrm{if }n\equiv0[2]\\
\end{eqnarray*}

The Jacobi $\theta$-series is defined for $\tau\in\mathbb{H}_1,z\in\mathbb{C}$:
\begin{eqnarray*}
\theta(\tau,z)&=&\sum_{n\in\mathbb{Z},n\equiv1[2]}(-1)^\frac{n-1}{2}e^{i\pi(\frac{n^2}{4}\tau+nz)}\\
&=&\sum\left(\frac{-4}{n}\right)q^{n^2/8}r^{n/2},\qquad q=e^{2i\pi\tau},\, r= e^{2i\pi z}\\
&=&-q^{1/8}r^{-1/2}\prod_{n\geq1}(1-q^{n-1}r)(1-q^nr^{-1})(1-q^n)
\end{eqnarray*}

\begin{thebibliography}{1}
\bibitem{EZ} Martin Eichler, Don Zagier: The theory of Jacobi Forms. Birkhauser, 1985
\bibitem{SerreCA}Jean-Pierre Serre Cours d'arithmetique. 4eme edition 1994. P.U.F.
\end{thebibliography}
\end{document}